\chapter{Resultados}

Para o desenvolvimento do simulador, foi escolhida a linguagem de programação Java. A motivação da escolha do Java se deve a diversos fatores:

\begin{itemize}
	\item Java é uma das linguagens mais utilizadas no momento
	\item A \textit{engine} jbox2D possui a parte de simulação de fluídos o que facilita trabalhos futuros na simulação de experiências com fluídos.
	\item Portabilidade dos arquivos executáveis do Java em qualquer Sistema Operacional, incluindo Android, o que facilita a utilização pelos alunos, professores e escolas.
	\item Ferramentas para construção da interface gráfica como o \textit{SceneBuilder}.
	\item Vasta documentação a respeito da linguagem.
\end{itemize}

Com a escolha do Java, o passo seguinte foi a escolha do pacote para a construção da interface. A escolhida foi o JavaFX que é um conjunto de pacotes gráficos e de mídia desenvolvidos pela Oracle e que permite os desenvolvedores criar aplicações de interfaces complexas para várias plataformas \url{http://docs.oracle.com/javase/8/javafx/get-started-tutorial/jfx-overview.htm#JFXST784}. Além disso o JavaFX também possui um pacote para a construção de gráficos que é um ponto importante para a apresentação dos resultados da simulação.

O JavaFX não foi apresentado durante o curso de Engenharia da Computação. Dessa forma foi necessário estudar o pacote, para a criação da interface do simulador e dos objetos que seriam simulados. Para o estudo foi utilizada a documentação presente no site da Oracle \url{http://docs.oracle.com/javase/8/javase-clienttechnologies.htm}. Para \textit{engine} jbox2D a documentação é escassa e não possui um manual de utilização. Dessa forma, foi necessário a leitura do manual da engine box2D que foi escrita para C++ já que a jbox2D é baseada nesta. 

O grande desafio deste trabalho foi integrar os resultados calculados pela \textit{engine} com a parte visual que está sendo implementada pelo JavaFX. Os seguintes problemas tiveram que ser tratados:

\begin{itemize}
	\item As coordenadas de espaço que são utilizadas na jbox2D são em metros já no JavaFX as coordenadas são em pixels. Sendo assim, foi necessário a implementação de funções para traduzir essas informações entre as bibliotecas.
	\item A unidade de utilizada para medir ângulo e o sentido também é diferente entre JavaFX - graus e a jbox2D - radianos. E o sentido de mudança angular é oposto, um é horário e o outro é anti-horário. Outras funções também foram utilizadas para solução deste problema.
	\item Além das unidades que diferem o ponto do corpo que é levado em consideração para o posicionamento entre as duas bibliotecas também diverge. Para o JavaFX o ponto utilizado é o vértice mais esquerda e mais alto para formas que não são circulares, para a jbox2D é utilizado o centro do corpo como referência. Portanto essa diferença entre os referenciais de posicionamento também foram tratadas.
	\item Para que as coordenadas sejam transformadas entre as bibliotecas é necessário que as dimensões do quadro que a simulação está sendo exibida sejam passadas como parâmetro e também as dimensões em metros do mundo que está sendo simulado pela jbox2D. A proporção entre largura e altura em \textit{pixels} deve ser a mesma proporção entre largura e altura do mundo criado pelo jbox2D.
\end{itemize}

Depois resolvidos os problemas, todas essas funções de conversão entre as coordenadas dos dois pacotes foram condensadas em uma classe chamada \textit{World2}. \textit{World2} é uma extensão da classe \textit{World} da \textit{jbox2D}. Conforme os objetivos desse trabalho, a classe\textit{ World2} serve de base para a criação do simulador onde os futuros desenvolvedores podem utiliza-la de maneira objetiva sem a necessidade de preocupação com os problemas citados acima.

\section{Simulador}
Concluída a parte de integração, é a hora de implementar a interface gráfica do simulador. A interface foi montada de acordo com \citeonline{RevBibConcepcao} ver figura \ref{res_figura1}.  O simulador foi dividido em quatro partes: Área de controle da simulação, Área de adição de objetos, Área de ajustes de objetos e finalmente a Área de simulação.

\begin{figure}[!htb]
	\caption{Tela Inicial do Simulador}
	\centering
	\includegraphics[scale=0.4]{Imagens/telaSim.png}
	
	\legend{Fonte: \url{http://box2d.org/manual.pdf}}
	\label{res_figura1}
\end{figure}


\section{Área de Simulação}
Na parte destacada da figura \ref{res_figura2} se encontra a área de simulação. É o mundo a ser simulado, possui gravidade que pode ser ajustada clicando na aba de Ajuste do Mundo. Além da gravidade é possível também alterar o tamanho do mundo que está sendo visualizado nesse quadro, o usuário pode definir sua altura ou largura também na aba Ajuste do Mundo. 
A Área de simulação é um quadro em branco no qual o professor pode montar o experimento que desejar para ser visualizado durante sua aula. É nessa área que ocorre a visualização dos fenômenos físicos estudados. É possível interagir com  os objetos dessa área através do mouse, o objeto pode ser selecionado ou reposicionado apenas clicando e arrastando para a posição desejada. 

\begin{figure}[!htb]
	\caption{Área de Simulação}
	\centering
	\includegraphics[scale=0.4]{Imagens/telaSim.png}
	
	\legend{Fonte: \url{}}
	\label{res_figura2}
\end{figure}


\section{Área de Controle da Simulação}
É o setor selecionado em vermelho na figura \ref{res_figura3}. Nesse setor é possível controlar a simulação das seguintes formas:

\begin{itemize}
	\item Inciar a simulação, utilizando o botão Iniciar.
	\item Parar a simulação, utilizando o botão Parar.
	\item Salvar simulação, utilizando o botão Salvar, é possível salvar toda a configuração da simulação para que seja retomada em qualquer momento desejado, essa função pode ser utilizada pelo professor criar um exemplo desejado para ser utilizado durante a sua aula. É gerado um arquivo que pode ser salvo no local escolhido pelo professor, de acordo com a figura \ref{salvar}.
	\item Carregar Simulação, utilizando o botão Carregar Simulação, é possível carregar uma simulação previamente preparada pelo próprio usuário ou por um usuário diferente de acordo com a figura \ref{carregar}, dessa forma, o professor pode compartilhar a simulação com os alunos ou passa-la como algum tipo de exercício ou esse arquivo pode ser compartilhado entre os próprios professores.
	\item reiniciar, através do botão Reiniciar, a simulação retorna para o estado inicial no momento em que o arquivo foi carregado.
	\item Tempo de Simulação, há um \textit{label} no canto direito que indica o tempo decorrido durante a simulação.
\end{itemize}

\section{Área de Adição de Figuras}
Área circulada em vermelho da figura \ref{} nesse parte do programa é onde o usuário pode escolher os objetos que desejar adicionar a simulação há 6 tipos de objetos:
\begin{itemize}
	\item Parede, é o item 1 da figura \ref{} o objeto parede serve para conter os objetos que so movimentam dentro do quadro de simulação. A parede é imóvel.
	\item Chão, é o item 2 da figura \ref{} pode ser usado como um limite de espaço inferior da simulação, ele é imóvel.
	\item Plano inclinado, é o item 3 da figura \ref{}  é utilizado para simulações de plano inclinado em que os objetos podem rolar sobre o plano e tem seu vetor de velocidade governado pela inclinação do plano. Também é imóvel.
	\item Circulo, é o item 4 da figura \ref{} é um corpo móvel em que as leis da física são aplicadas, é ele quem vai ser mover durante a simulação.
	\item Retângulo, é o item 5 da figura \ref{} é um corpo móvel em que as leis da física são aplicadas, é ele quem vai ser mover durante a simulação.
	\item Sensor, é o item 6 da figura \ref{} Ele não interage fisicamente com os objetos da simulação. Ele é utilizado para parar a simulação quando o objeto encosta nele ou quando o objeto passa completamente por ele. Isso pode ser utilizado para visualizar as variáveis desejadas por exemplo quando um objeto cai e acerta o chão.
\end{itemize}

\begin{figure}[!htb]
	\caption{Área de Simulação}
	\centering
	\includegraphics[scale=0.4]{Imagens/telaSim.png}
	
	\legend{Fonte: \url{}}
	\label{res_figura3}
\end{figure}

As classe fundamentais da engine são: World, Body, BodyDef, FixtureDef.

\section{Classes}

A classe World é responsável pelo controle da simulação. É através dela que são feitos os cálculos da evolução da simulação. A evolução consiste no cálculo da posição dos corpos criados no mundo que está sendo simulado e a interação entre os corpos.
